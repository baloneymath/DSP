\documentclass{article}

\usepackage{fancyhdr}
\usepackage{extramarks}
\usepackage{amsmath}
\usepackage{amsthm}
\usepackage{amssymb}
%\usepackage{fontspec}
\usepackage{amsfonts}
%\usepackage{xeCJK}
\usepackage{algpseudocode}
\usepackage{listings}
%%plots
\usepackage{tikz}
\usetikzlibrary{automata,positioning}
\usepackage{graphicx}
\usepackage{pgfplots}
\pgfplotsset{compat=newest} % Allows to place the legend below plot
\usepgfplotslibrary{units} % Allows to enter the units nicely


%
% Basic Document Settings
%

\topmargin=-0.45in
\evensidemargin=0in
\oddsidemargin=0in
\textwidth=6.5in
\textheight=9.0in
\headsep=0.25in

\linespread{1.1}

\pagestyle{fancy}
\lhead{\AuthorName}
\chead{\Class\ (\ClassInstructor\ \ClassTime): \Title}
\rhead{\firstxmark}
\lfoot{\lastxmark}
\cfoot{\thepage}

\renewcommand\headrulewidth{0.4pt}
\renewcommand\footrulewidth{0.4pt}

\setlength\parindent{0pt}

\newcommand{\Title}{Report \#3}
\newcommand{\FinishDate}{\today}
\newcommand{\Class}{Digital Speech Processing}
\newcommand{\ClassTime}{}
\newcommand{\ClassInstructor}{Professor Lin-Shan Lee}
\newcommand{\Department}{EE2}
\newcommand{\AuthorID}{b03901016}
\newcommand{\AuthorName}{Hao Chen}

%util
\newcommand{\horline}[1]{\rule{\linewidth}{#1}} % Create horizontal rule command with 1 argument of height
\newcommand\n{\mbox{\qquad}}

%
% Title Page
%

\title{
{National Taiwan University}\\    
    \textmd{\textbf{\Class:\ \Title}}
}

\author{
	\Department \ \AuthorID \\
	\textbf{\AuthorName}
}
\date{
	\today \\
	\horline{1pt}
}



\renewcommand{\part}[1]{\textbf{\large Part \Alph{partCounter}}\stepcounter{partCounter}\\}

%
% Various Helper Commands
%

% Useful for algorithms
\newcommand{\alg}[1]{\textsc{\bfseries \footnotesize #1}}

% For derivatives
\newcommand{\deriv}[1]{\frac{\mathrm{d}}{\mathrm{d}x} (#1)}

% For partial derivatives
\newcommand{\pderiv}[2]{\frac{\partial}{\partial #1} (#2)}

% Integral dx
\newcommand{\dx}{\mathrm{d}x}
% Alias for the Solution section header
\newcommand{\solution}{\textbf{\large Solution}}

% Probability commands: Expectation, Variance, Covariance, Bias
\newcommand{\E}{\mathrm{E}}
\newcommand{\Var}{\mathrm{Var}}
\newcommand{\Cov}{\mathrm{Cov}}
\newcommand{\Bias}{\mathrm{Bias}}


\begin{document}
\maketitle
\section{Introduciton}
\n This program aim to fix sentences which contains ZhuYin to the most suitable sentences contains only Chinese words, based on the related language model.

\section{Environment}
\begin{itemize}
	\item OS : Linux mint 17.2 Rafaela
	\item Machine Type : i686-m64
	\item Kernel : x86\_64 Linux 3.16.0-38-generic
	\item Shell : zsh 5.0.2
\end{itemize}

\section{Usage}
\begin{itemize}
	\item \large{How to complie :} \\
	\normalsize At the path includes \texttt{Makefile, bigram.lm, Big5-ZhuYin.map, mapping.cpp, mydisambig.cpp} \\
		\n 1. \texttt{make clean:} Initialization\\
		\n 2. \texttt{make map:} create \texttt{ZhuYin-Big5.map} from \texttt{Big5-ZhuYin.map} \\
		\n 3. \texttt{make:} Build \texttt{mydisambig}
		
	\item \large{How to execute :} \\
	\normalsize First create \texttt{ZhuYin-Big5.map},\\
	\underline{ex.}\\
	\texttt{./mapping Big5-ZhuYin.map ZhuYin-Big5.map}\\
	where \texttt{argv[1]} is the input and \texttt{argv[2]} is the output.
	\normalsize 
	\\ \\At the path includes \texttt{testdata/}, simply type \texttt{make run} to run through all the files in \texttt{testdata/}, and check all the outputs at \texttt{result2/}. \\
	For more details about \texttt{mydisambig}, as an example, \\
	\underline{ex.}\\
	\n \texttt{./mydisambig testdata.txt output.txt}\\
	where \texttt{argv[1]} is the input testdata, \texttt{argv[2]} is the output file.
\end{itemize}
\end{document}
