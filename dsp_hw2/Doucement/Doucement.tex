\documentclass{article}

\usepackage{fancyhdr}
\usepackage{extramarks}
\usepackage{amsmath}
\usepackage{amsthm}
\usepackage{amssymb}
%\usepackage{fontspec}
\usepackage{amsfonts}
%\usepackage{xeCJK}
\usepackage{algpseudocode}
\usepackage{listings}
%%plots
\usepackage{tikz}
\usetikzlibrary{automata,positioning}
\usepackage{graphicx}
\usepackage{pgfplots}
\pgfplotsset{compat=newest} % Allows to place the legend below plot
\usepgfplotslibrary{units} % Allows to enter the units nicely


%
% Basic Document Settings
%

\topmargin=-0.45in
\evensidemargin=0in
\oddsidemargin=0in
\textwidth=6.5in
\textheight=9.0in
\headsep=0.25in

\linespread{1.1}

\pagestyle{fancy}
\lhead{\AuthorName}
\chead{\Class\ (\ClassInstructor\ \ClassTime): \Title}
\rhead{\firstxmark}
\lfoot{\lastxmark}
\cfoot{\thepage}

\renewcommand\headrulewidth{0.4pt}
\renewcommand\footrulewidth{0.4pt}

\setlength\parindent{0pt}

\newcommand{\Title}{Report \#2}
\newcommand{\FinishDate}{\today}
\newcommand{\Class}{Digital Speech Processing}
\newcommand{\ClassTime}{}
\newcommand{\ClassInstructor}{Professor Lin-Shan Lee}
\newcommand{\Department}{EE2}
\newcommand{\AuthorID}{b03901016}
\newcommand{\AuthorName}{Hao Chen}

%util
\newcommand{\horline}[1]{\rule{\linewidth}{#1}} % Create horizontal rule command with 1 argument of height
\newcommand\n{\mbox{\qquad}}

%
% Title Page
%

\title{
{National Taiwan University}\\    
    \textmd{\textbf{\Class:\ \Title}}
}

\author{
	\Department \ \AuthorID \\
	\textbf{\AuthorName}
}
\date{
	\today \\
	\horline{1pt}
}



\renewcommand{\part}[1]{\textbf{\large Part \Alph{partCounter}}\stepcounter{partCounter}\\}

%
% Various Helper Commands
%

% Useful for algorithms
\newcommand{\alg}[1]{\textsc{\bfseries \footnotesize #1}}

% For derivatives
\newcommand{\deriv}[1]{\frac{\mathrm{d}}{\mathrm{d}x} (#1)}

% For partial derivatives
\newcommand{\pderiv}[2]{\frac{\partial}{\partial #1} (#2)}

% Integral dx
\newcommand{\dx}{\mathrm{d}x}
% Alias for the Solution section header
\newcommand{\solution}{\textbf{\large Solution}}

% Probability commands: Expectation, Variance, Covariance, Bias
\newcommand{\E}{\mathrm{E}}
\newcommand{\Var}{\mathrm{Var}}
\newcommand{\Cov}{\mathrm{Cov}}
\newcommand{\Bias}{\mathrm{Bias}}


\begin{document}
\maketitle
\section{Introduciton}
\n Using HTK toolkit (HCompV, HCopy, HHed, HERest...) to build the HMM models of continuous signal and test the accuracy. 

\section{Environment}
\begin{itemize}
	\item OS : Linux mint 17.2 Rafaela
	\item Kernel : x86\_64 Linux 3.16.0-38-generic
	\item Shell : zsh 5.0.2
\end{itemize}

\section{Steps}
\begin{enumerate}
	\item Run all shellscripts \\(
		\texttt{
			 00\_clean\_all.sh,
			 01\_run\_HCopy.sh,
			 02\_run\_HCompV.sh,
			 03\_training.sh,
			 04\_testing.sh
		} )
	\item Check \texttt{accuracy} in \texttt{/result} if it's \texttt{74.34}
	\item Change the data in "\texttt{proto}" to increase states and
		"\texttt{mix2\_10.hed}" to increase Gaussian mixtures.
	\item Maximize the accuracy.
\end{enumerate}

\section{Result}
\n After running all the five shellscripts, I just change the number of states to 10, and increase the number of Gaussian mixtures to 12, then I get the accuracy "\texttt{95.97}".
\\
\n I found that the number of iteration does not effect a lot, so I didn't optimize it.
\end{document}
